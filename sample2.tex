\documentclass{ltjsarticle}

\usepackage{graphics}
\begin{document}

\title{LilyPond-Bookによる楽典資料生成試験}
\author{haru}
\maketitle

\tableofcontents
\clearpage

\section{音符の種類}

音符にはその長さ(「音価」という)によって複数の種類が存在する。\\

{%
\parindent 0pt
\noindent
\ifx\preLilyPondExample \undefined
\else
  \expandafter\preLilyPondExample
\fi
\def\lilypondbook{}%
\input{16/lily-7e667693-systems.tex}%
\ifx\postLilyPondExample \undefined
\else
  \expandafter\postLilyPondExample
\fi
}

上記の譜面は、譜面上の「ソ」の位置に音符をプロットした例である。 \\
左から順に、全音符、二分音符、四分音符、八分音符、十六分音符、三十二分音符である。

\end{document}
